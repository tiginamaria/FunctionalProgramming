\documentclass[10pt]{report}
\usepackage[utf8]{inputenc}
\usepackage[russian]{babel}
\usepackage[centertags]{amsmath}
\usepackage{amsfonts}
\usepackage{amssymb}
\usepackage{dsfont}
\usepackage{euler}
\usepackage{eulervm}
\usepackage{graphics}
\usepackage[all,cmtip]{xy}
%\usepackage{diagxy}
\usepackage{cmap}
\usepackage[T2A]{fontenc}
\usepackage[utf8]{inputenc}
\usepackage[russian]{babel}
\usepackage{graphicx}
\usepackage{amsthm,amsmath,amssymb}
\usepackage[russian,colorlinks=true,urlcolor=red,linkcolor=blue]{hyperref}
\usepackage{enumerate}
\usepackage{datetime}
\usepackage{minted}
\usepackage{fancyhdr}
\usepackage{lastpage}
\usepackage{color}
\usepackage{verbatim}
\usepackage{tikz}


\parskip=0em
\parindent=0em

\sloppy
\voffset=-20mm
\textheight=235mm
\hoffset=-25mm
\textwidth=180mm
\headsep=12pt
\footskip=20pt

\setcounter{page}{0}
\pagestyle{empty}

% Основные математические символы
\DeclareSymbolFont{extraup}{U}{zavm}{m}{n}
\DeclareMathSymbol{\heart}{\mathalpha}{extraup}{86}
\def\TODO{{\color{red}\bf TODO}}
\def\N{\mathbb{N}}       %
\def\R{\mathbb{R}}       %
\def\Z{\mathbb{Z}}       %
\def\INF{\t{+}\infty}    % +inf
\def\EPS{\varepsilon}    %
\def\EMPTY{\varnothing}  %
\def\PHI{\varphi}        %
\def\ar{\rightarrow}     % ->
\def\l{\backslash}       % \
\def\SO{\Rightarrow}     % =>
\def\EQ{\Leftrightarrow} % <=>
\def\t{\texttt}          % mono font
\def\O{\mathcal{O}}      %
\def\NO{\t{\#}}          % #
\def\XOR{\text{ {\raisebox{-2pt}{\ensuremath{\Hat{}}}} }}
\def\letus{%
    \mathord{\setbox0=\hbox{$\exists$}%
             \hbox{\kern 0.125\wd0%
                   \vbox to \ht0{%
                      \hrule width 0.75\wd0%
                      \vfill%
                      \hrule width 0.75\wd0}%
                   \vrule height \ht0%
                   \kern 0.125\wd0}%
           }%
}
\renewcommand{\le}{\leqslant}
\renewcommand{\ge}{\geqslant}
\newcommand{\q}[1]{\langle #1 \rangle}               % <x>
\newcommand\URL[1]{{\footnotesize{\url{#1}}}}        %
\newcommand{\sfrac}[2]{{\scriptstyle\frac{#1}{#2}}}  % Очень маленькая дробь
\newcommand{\mfrac}[2]{{\textstyle\frac{#1}{#2}}}    % Небольшая дробь

\newcommand{\fix}[1]{{\color{red}#1}}

\newcommand{\score}[1]{{\bf\color{red}{(#1)}}}

% Отступы
\def\makeparindent{\hspace*{\parindent}}
\def\up{\vspace*{-0.3em}}
\def\down{\vspace*{0.3em}}
\def\LINE{\vspace*{-1em}\noindent \underline{\hbox to 1\textwidth{{ } \hfil{ } \hfil{ } }}}
%\def\up{\vspace*{-\baselineskip}}

\lfoot{}
\cfoot{\thepage\t{/}\pageref*{LastPage}}
\rfoot{}
\renewcommand{\footrulewidth}{0.4pt}

\newenvironment{MyList}[1][4pt]{
  \begin{enumerate}[1.]
  \setlength{\parskip}{0pt}
  \setlength{\itemsep}{#1}
}{       
  \end{enumerate}
}

\newenvironment{MyList0}[1][4pt]{
  \begin{enumerate}[0.]
  \setlength{\parskip}{0pt}
  \setlength{\itemsep}{#1}
}{       
  \end{enumerate}
}

\newenvironment{InnerMyList}[1][0pt]{
  \vspace*{-0.5em}
  \begin{enumerate}[a)]
  \setlength{\parskip}{#1}
  \setlength{\itemsep}{0pt}
}{
  \end{enumerate}
}

\newcommand{\Section}[1]{
  \refstepcounter{section}
  \addcontentsline{toc}{section}{\arabic{section}. #1} 
  %{\LARGE \bf \arabic{section}. #1} 
  {\LARGE \bf #1} 
  \vspace*{1em}
  \makeparindent\unskip
}
\newcommand{\Subsection}[1]{
  \refstepcounter{subsection}
  \addcontentsline{toc}{subsection}{\arabic{section}.\arabic{subsection}. #1} 
  {\Large \bf \arabic{section}.\arabic{subsection}. #1} 
  \vspace*{1em}
  \makeparindent\unskip
}

\newcommand{\lra}{\Longleftrightarrow}
\newcommand{\ra}{\Longrightarrow}
\newcommand{\ds}{\displaystyle}



\newcommand{\ta}{\!\rightarrow\!}
\newcommand{\tr}{\!:\!}
\newcommand{\Tr}{:}

% канонические комбинаторы Y, S, K, I, etc
\newcommand{\canonComb}[1]{\boldsymbol{#1}}
% обычные комбинаторы fac, mult, etc
\newcommand{\comb}[1]{\mathtt{#1}} %% \mathsf

\newenvironment{MyList}[1][4pt]{
  \begin{enumerate}[0.]
  \setlength{\parskip}{0pt}
  \setlength{\itemsep}{#1}
}{       
  \end{enumerate}
}



\begin{document}

\thispagestyle{empty}

\begin{center}
\textbf{Курс: Функциональное программирование}

\textbf{Домашнее задание 3}

\textbf{Тигиной Марии 2 группа}
\end{center}

\bigskip

\bigskip




%-------------------------------------------------------------------------------------


\begin{MyList}[8pt]

\item Заселите типы:
\begin{enumerate}[a)]
\item $a \ar b \ar c \ar a$\\ 
$\l a\; b\; c \ar a$\\
(1 балл)
\item $a \ar b \ar c \ar b$\\
$\l a\; b\; c \ar b$\\
(1 балл)
\item $a \ar b \ar a \ar a$\\
$\l a\;b\;c \ar c$\\
$\l a\;b\;c \ar a$\\
(1 балл)
\item $a \ar a \ar a \ar a$\\
$\l a\;b\;c \ar a$\\
$\l a\;b\;c \ar b$\\
$\l a\;b\;c \ar c$\\
(1 балл)
\end{enumerate}

\item Найдите обитателей типов:
\begin{enumerate}[a)]
\item $(d  \ar d  \ar  a)  \ar  (a  \ar  b  \ar  c)  \ar  (d  \ar  b)  \ar  d  \ar  c$\\
$f :\;d  \ar d  \ar  a$\\
$g :\;a  \ar  b  \ar  c$\\
$h :\;d  \ar  b$\\
$\l f\;g\;h\;x \ar g\; (f\; x\; x) (h\;x)$\\
(1,5 балла)

\item $(d \ar d \ar a) \ar (c \ar a) \ar (a \ar b) \ar d \ar c \ar b$\\
$f :\;d  \ar d  \ar  a$\\
$g :\;c  \ar  a$\\
$h :\;a  \ar  b$\\
$\l f\;g\;h\;x\;y \ar h\; (f\; x\; x)$\\
(1,5 балла)\\
Сколько обитателей второго типа вы можете привести?\\
Как минимум еще один:\\
$\l f\;g\;h\;x\;y \ar h\; (g\; y)$\\
\end{enumerate}\\


\itemТипизируйте по Чёрчу :
\begin{enumerate}[a)]
\item $\canonComb{S}\,\canonComb{K}\,\canonComb{K}$\\
$(\l f^{a \ar (b \ar a) \ar a}\; g^{a \ar b \ar a}\; x^{a} \ar f\; x\; (g\; x))(\l x^a\;y^{b\ar a} \ar x)(\l x^a\;y^b \ar x) \; :\; a \ar a$\\
(1 балл)
\item $\canonComb{S}\,\canonComb{K}\,\canonComb{I}$\\
$(\l f^{a \ar a \ar a}\; g^{a \ar a}\; x^{a} \ar f\; x\; (g\; x))(\l x^a\;y^a\ar a} \ar x)(\l x^a \ar x) \; :\; a \ar a$\\
(1 балл)
\end{enumerate}

\item Сконструируйте терм типа:\\
$(c \ar e) \ar ((c \ar e) \ar e) \ar e$\\
Которому нельзя было бы приписать тип:\\
$a \ar (a \ar e) \ar e$\\
$\l f^{c\ar e}\; g^{(c\ar e) \ar e}\; \ar g(\l c \ar gf)$\\
(2 балла)

\item Сконструируйте терм типа:
\begin{enumerate}[a)]
\item $ ((a \ar b) \ar a) \ar (a \ar a \ar b) \ar a$ \\
$\l f\; g\; \ar f\;(\l a \ar g\; a\;a)$\\
(2 балла)
\item $ ((a \ar b) \ar a) \ar (a \ar a \ar b) \ar b$\\
$\l f\; g\; \ar g\; (f\;(\l a \ar g\; a\;a))\; (f\;(\l a \ar g\; a\;a))$\\
(2 балла)
\item $ ((((a \ar b) \ar a) \ar a) \ar b) \ar b$ \\
$p \;:\;a \ar b$\\
$q\;:\;(a \ar b) \ar a$\\
$q = \l p \ar (\l a \ar K\;a\;(p\;a)\;)$\\
$h\;:\;((a \ar b) \ar a) \ar a$\\
$h = \l q \ar(\l p \ar (\l a \ar K\;a\;(p(q\;p))\;))$\\
$g\;:\;(((a \ar b) \ar a) \ar a) \ar b$\\
$g = \l h \ar( \l q \ar(\l p \ar (\l a \ar K\; (p(h\;q))\;(p(q\;p))\;)))$\\
$f\;:\;((((a \ar b) \ar a) \ar a) \ar b)\ar b$\\
$f = \l g \ar (\l h \ar( \l q \ar(\l p \ar (\l a \ar K\; (g\;h)\;\; (K\;(p(h\;q))\;(K\; (p\;a) \;(p(q\;p))\;)\;)\;))))$

Можно попробовать ещё так:\\
$ ((((a \ar b) \ar a) \ar a) \ar b) \ar b$ \\
$p \;:\;a \ar (a \ar a) = a \ar b$\\
$q\;:\;(a \ar b) \ar a$\\
$q = \l p \ar (\l a \ar (p\;a)\; a)$\\
$h\;:\;((a \ar b) \ar a) \ar a$\\
$h = \l q \ar(\l p \ar (\l a \ar (p(q\;p))((p\;a)a)\;))$\\
$g\;:\;(((a \ar b) \ar a) \ar a) \ar b$\\
$g = \l h \ar( \l q \ar(\l p \ar (\l a \ar p((p(h\;q))((p(q\;p))((p\;a)a))))))$\\
$f\;:\;((((a \ar b) \ar a) \ar a) \ar b)\ar b$\\
$f = \l g \ar (\l h \ar( \l q \ar(\l p \ar (\l a \ar (g\; h)((p(h\;q))((p(q\;p))((p\;a)a)))))))$\\
(3 балла)
\end{enumerate}\\

\end{MyList0}
\end{document}
