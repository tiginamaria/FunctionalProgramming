\documentclass[10pt]{report}
\usepackage[utf8]{inputenc}
\usepackage[russian]{babel}
\usepackage[centertags]{amsmath}
\usepackage{amsfonts}
\usepackage{amssymb}
\usepackage{dsfont}
\usepackage{euler}
\usepackage{eulervm}
\usepackage{graphics}
\usepackage[all,cmtip]{xy}
%\usepackage{diagxy}
\usepackage{cmap}
\usepackage[T2A]{fontenc}
\usepackage[utf8]{inputenc}
\usepackage[russian]{babel}
\usepackage{graphicx}
\usepackage{amsthm,amsmath,amssymb}
\usepackage[russian,colorlinks=true,urlcolor=red,linkcolor=blue]{hyperref}
\usepackage{enumerate}
\usepackage{datetime}
\usepackage{minted}
\usepackage{fancyhdr}
\usepackage{lastpage}
\usepackage{color}
\usepackage{verbatim}
\usepackage{tikz}


\parskip=0em
\parindent=0em

\sloppy
\voffset=-20mm
\textheight=235mm
\hoffset=-25mm
\textwidth=180mm
\headsep=12pt
\footskip=20pt

\setcounter{page}{0}
\pagestyle{empty}

% Основные математические символы
\DeclareSymbolFont{extraup}{U}{zavm}{m}{n}
\DeclareMathSymbol{\heart}{\mathalpha}{extraup}{86}
\def\TODO{{\color{red}\bf TODO}}
\def\N{\mathbb{N}}       %
\def\R{\mathbb{R}}       %
\def\Z{\mathbb{Z}}       %
\def\INF{\t{+}\infty}    % +inf
\def\EPS{\varepsilon}    %
\def\EMPTY{\varnothing}  %
\def\PHI{\varphi}        %
\def\ar{\rightarrow}     % ->
\def\l{\backslash}       % \
\def\SO{\Rightarrow}     % =>
\def\EQ{\Leftrightarrow} % <=>
\def\t{\texttt}          % mono font
\def\O{\mathcal{O}}      %
\def\NO{\t{\#}}          % #
\def\XOR{\text{ {\raisebox{-2pt}{\ensuremath{\Hat{}}}} }}
\renewcommand{\le}{\leqslant}
\renewcommand{\ge}{\geqslant}
\newcommand{\q}[1]{\langle #1 \rangle}               % <x>
\newcommand\URL[1]{{\footnotesize{\url{#1}}}}        %
\newcommand{\sfrac}[2]{{\scriptstyle\frac{#1}{#2}}}  % Очень маленькая дробь
\newcommand{\mfrac}[2]{{\textstyle\frac{#1}{#2}}}    % Небольшая дробь

\newcommand{\fix}[1]{{\color{red}#1}}

\newcommand{\score}[1]{{\bf\color{red}{(#1)}}}

% Отступы
\def\makeparindent{\hspace*{\parindent}}
\def\up{\vspace*{-0.3em}}
\def\down{\vspace*{0.3em}}
\def\LINE{\vspace*{-1em}\noindent \underline{\hbox to 1\textwidth{{ } \hfil{ } \hfil{ } }}}
%\def\up{\vspace*{-\baselineskip}}

\lfoot{}
\cfoot{\thepage\t{/}\pageref*{LastPage}}
\rfoot{}
\renewcommand{\footrulewidth}{0.4pt}

\newenvironment{MyList}[1][4pt]{
  \begin{enumerate}[1.]
  \setlength{\parskip}{0pt}
  \setlength{\itemsep}{#1}
}{       
  \end{enumerate}
}

\newenvironment{MyList0}[1][4pt]{
  \begin{enumerate}[0.]
  \setlength{\parskip}{0pt}
  \setlength{\itemsep}{#1}
}{       
  \end{enumerate}
}

\newenvironment{InnerMyList}[1][0pt]{
  \vspace*{-0.5em}
  \begin{enumerate}[a)]
  \setlength{\parskip}{#1}
  \setlength{\itemsep}{0pt}
}{
  \end{enumerate}
}

\newcommand{\Section}[1]{
  \refstepcounter{section}
  \addcontentsline{toc}{section}{\arabic{section}. #1} 
  %{\LARGE \bf \arabic{section}. #1} 
  {\LARGE \bf #1} 
  \vspace*{1em}
  \makeparindent\unskip
}
\newcommand{\Subsection}[1]{
  \refstepcounter{subsection}
  \addcontentsline{toc}{subsection}{\arabic{section}.\arabic{subsection}. #1} 
  {\Large \bf \arabic{section}.\arabic{subsection}. #1} 
  \vspace*{1em}
  \makeparindent\unskip
}

\newcommand{\lra}{\Longleftrightarrow}
\newcommand{\ra}{\Longrightarrow}
\newcommand{\ds}{\displaystyle}



\newcommand{\ta}{\!\rightarrow\!}
\newcommand{\tr}{\!:\!}
\newcommand{\Tr}{:}

% канонические комбинаторы Y, S, K, I, etc
\newcommand{\canonComb}[1]{\boldsymbol{#1}}
% обычные комбинаторы fac, mult, etc
\newcommand{\comb}[1]{\mathtt{#1}} %% \mathsf

\newenvironment{MyList}[1][4pt]{
  \begin{enumerate}[0.]
  \setlength{\parskip}{0pt}
  \setlength{\itemsep}{#1}
}{       
  \end{enumerate}
}


\begin{document}

\thispagestyle{empty}

\begin{center}
\textbf{Курс: Функциональное программирование}

\textbf{Домашнее задание 1}

\textbf{Тигиной Марии 2 группа}
\end{center}

\bigskip

\bigskip




%-------------------------------------------------------------------------------------


\begin{MyList0}[8pt]

\item Покажите, что:
\begin{enumerate}[a)]
\item \canonComb{I}=\canonComb{S}\,\canonComb{K}\,\canonComb{K}\\
$\canonComb{S}\,\canonComb{K}\,\canonComb{K}\\
=(\l xyz \ar xz(yz))KK \\
= \l yz \ar Kz(yz) K \\
= \l z \ar Kz(Kz)\\
= \l z \ar (\l xy \ar x)z(Kz)\\
= \l z \ar z\\
= \canonComb{I}$

(1 балл)\\

\item \canonComb{K^*}=\canonComb{K}\,\canonComb{I} \\
$\canonComb{K}\,\canonComb{I}\\
= (\l xy \ar x)(\l z \ar z)\\
= (\l y \ar (\l z \ar z))\\
= (\l y z \ar z)\\
=\canonComb{K^*}$

(1 балл)\\
\end{enumerate}
\end{MyList0}

\begin{MyList}[8pt]

\item Выделите свободные и связанные переменные в термах и осуществите подстановки:\\
(одним цветом выделены связанные вершины)
\begin{enumerate}[a)]

\item $x (\l xy \ar y (x w) u) y$  $[x := \l z \ar z]$\\
$x(\l \textcolor{blue}{x} \textcolor{red}{y} \ar \textcolor{red}{y} (\textcolor{blue}{x} w) u) y$  $[x := \l z \ar z]$\\
$(\l z \ar z) (\l \textcolor{blue}{x} \textcolor{red}{y} \ar \textcolor{red}{y} (\textcolor{blue}{x} w) u) y$\\
(1 балл)\\

\item $(\l x \ar x (\l y \ar y x) w) (\l x \ar v)$ $[w := y (\l v \ar v x)]$\\
$(\l \textcolor{blue}{x} \ar \textcolor{blue}{x} (\l \textcolor{red}{y} \ar \textcolor{red}{y} \textcolor{blue}{x}) w) (\l x \ar v)$ $[w := y (\l v \ar v x)]$\\
$(\l \textcolor{blue}{x} \ar \textcolor{blue}{x} (\l \textcolor{red}{y} \ar \textcolor{red}{y} \textcolor{blue}{x}) y (\l v \ar v x)) (\l x \ar v)$ \\
(1 балл)
\end{enumerate}


\item Уберите лишние скобки и осуществите бета-преобразование(если это
возможно):
\begin{enumerate}[a)]
\item $((\l x \ar (\l y \ar ((x y) z))) (a (b c)))$\\
$(\l x y \ar x y z) (a (b c))$\\
$\l y \ar a (b c) y z$\\
(1 балл)\\

\item $(((m n) m) (\l x \ar ((x (u v)) y))) $\\
$m n m (\l x \ar x (u v) y) $\\
(1 балл)\\

\item $((\l x \ar (\l y \ar ((y x) x))) (x (x y)) y)$\\
$(\l x \ar (\l y \ar y x x))(x (x y)) y$\\
$(\l y \ar y (x (x y)) (x (x y))) y$\\
$y (x (x y)) (x (x y))$\\
(1 балл)

\end{enumerate}

\item Покажите, что:\\

\canonComb{B}=\canonComb{S}\,(\canonComb{K}\,\canonComb{S})\,\canonComb{K}\\
$\canonComb{S}\,(\canonComb{K}\,\canonComb{S})\,\canonComb{K}\\
=(\l xyf \ar xz(yf))(KS)K\\
= \l f \ar (KS)f(Kf)\\
= \l f \ar KSf(Kf)\\
= \l f \ar (\l x y \ar x )Sf(Kf)\\
= \l f \ar (SKf)\\
= \l f \ar (\l hgx \ar hx(gx))(Kf)\\
= \l f \ar (\l gx \ar Kfx(gx))\\
= \l fgx \ar f(gx)\\
= \canonComb{B}$

(2 балла)

\item Напишите терм, ведущий себя как логическая связка xor\\
$XOR =$ $\l b1$ $ b2 \ar b1$ ($NOT$ $b2)$ $b2$ $\ar b1$ $(b2$ $FALSE$ $TRUE$)  $b2$

(1 балл)

\item Напишите терм, возводящий число Чёрча в степень(b – это
основание, а e – это показатель степени):\\
$POW=$$ \l b$ $e \ar e\; b\;$\\
(1 балла)

\item Докажите дистрибутивность сложения относительно умножения для чисел Чёрча\\
$PLUS = \l n m \ar n s(m s z)\\
MULT = \l n m \ar n (m s) z\\
MULT = \l n m \ar n (PLUS\; m) 0\\
MULT\; a\; (PLUS\; b\; c) = PLUS\; (MULT\; a\; b)\; (MULT\; a\; c)$\\
Достаточно посмотреть, какая кратность s( у результата этой опирации.\\
1) сначала к z применяется $c$ раз s(\\
2) затем еще $b$ раз, итого (b+c) раз s(\\
3) дальше это применение повторяется a раз, итого a(b+c) s(\\
4) так как a, b, c - целые, то a(b+c) - это тоже самое, что сначала применить ac, затем bc раз s(, что и есть формула для суммы произведений.



(2 балла)

\end{MyList0}
\end{document}
