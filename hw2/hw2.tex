\documentclass[10pt]{report}
\usepackage[utf8]{inputenc}
\usepackage[russian]{babel}
\usepackage[centertags]{amsmath}
\usepackage{amsfonts}
\usepackage{amssymb}
\usepackage{dsfont}
\usepackage{euler}
\usepackage{eulervm}
\usepackage{graphics}
\usepackage[all,cmtip]{xy}
%\usepackage{diagxy}
\usepackage{cmap}
\usepackage[T2A]{fontenc}
\usepackage[utf8]{inputenc}
\usepackage[russian]{babel}
\usepackage{graphicx}
\usepackage{amsthm,amsmath,amssymb}
\usepackage[russian,colorlinks=true,urlcolor=red,linkcolor=blue]{hyperref}
\usepackage{enumerate}
\usepackage{datetime}
\usepackage{minted}
\usepackage{fancyhdr}
\usepackage{lastpage}
\usepackage{color}
\usepackage{verbatim}
\usepackage{tikz}


\parskip=0em
\parindent=0em

\sloppy
\voffset=-20mm
\textheight=235mm
\hoffset=-25mm
\textwidth=180mm
\headsep=12pt
\footskip=20pt

\setcounter{page}{0}
\pagestyle{empty}

% Основные математические символы
\DeclareSymbolFont{extraup}{U}{zavm}{m}{n}
\DeclareMathSymbol{\heart}{\mathalpha}{extraup}{86}
\def\TODO{{\color{red}\bf TODO}}
\def\N{\mathbb{N}}       %
\def\R{\mathbb{R}}       %
\def\Z{\mathbb{Z}}       %
\def\INF{\t{+}\infty}    % +inf
\def\EPS{\varepsilon}    %
\def\EMPTY{\varnothing}  %
\def\PHI{\varphi}        %
\def\ar{\rightarrow}     % ->
\def\l{\backslash}       % \
\def\SO{\Rightarrow}     % =>
\def\EQ{\Leftrightarrow} % <=>
\def\t{\texttt}          % mono font
\def\O{\mathcal{O}}      %
\def\NO{\t{\#}}          % #
\def\XOR{\text{ {\raisebox{-2pt}{\ensuremath{\Hat{}}}} }}
\renewcommand{\le}{\leqslant}
\renewcommand{\ge}{\geqslant}
\newcommand{\q}[1]{\langle #1 \rangle}               % <x>
\newcommand\URL[1]{{\footnotesize{\url{#1}}}}        %
\newcommand{\sfrac}[2]{{\scriptstyle\frac{#1}{#2}}}  % Очень маленькая дробь
\newcommand{\mfrac}[2]{{\textstyle\frac{#1}{#2}}}    % Небольшая дробь

\newcommand{\fix}[1]{{\color{red}#1}}

\newcommand{\score}[1]{{\bf\color{red}{(#1)}}}

% Отступы
\def\makeparindent{\hspace*{\parindent}}
\def\up{\vspace*{-0.3em}}
\def\down{\vspace*{0.3em}}
\def\LINE{\vspace*{-1em}\noindent \underline{\hbox to 1\textwidth{{ } \hfil{ } \hfil{ } }}}
%\def\up{\vspace*{-\baselineskip}}

\lfoot{}
\cfoot{\thepage\t{/}\pageref*{LastPage}}
\rfoot{}
\renewcommand{\footrulewidth}{0.4pt}

\newenvironment{MyList}[1][4pt]{
  \begin{enumerate}[1.]
  \setlength{\parskip}{0pt}
  \setlength{\itemsep}{#1}
}{       
  \end{enumerate}
}

\newenvironment{MyList0}[1][4pt]{
  \begin{enumerate}[0.]
  \setlength{\parskip}{0pt}
  \setlength{\itemsep}{#1}
}{       
  \end{enumerate}
}

\newenvironment{InnerMyList}[1][0pt]{
  \vspace*{-0.5em}
  \begin{enumerate}[a)]
  \setlength{\parskip}{#1}
  \setlength{\itemsep}{0pt}
}{
  \end{enumerate}
}

\newcommand{\Section}[1]{
  \refstepcounter{section}
  \addcontentsline{toc}{section}{\arabic{section}. #1} 
  %{\LARGE \bf \arabic{section}. #1} 
  {\LARGE \bf #1} 
  \vspace*{1em}
  \makeparindent\unskip
}
\newcommand{\Subsection}[1]{
  \refstepcounter{subsection}
  \addcontentsline{toc}{subsection}{\arabic{section}.\arabic{subsection}. #1} 
  {\Large \bf \arabic{section}.\arabic{subsection}. #1} 
  \vspace*{1em}
  \makeparindent\unskip
}

\newcommand{\lra}{\Longleftrightarrow}
\newcommand{\ra}{\Longrightarrow}
\newcommand{\ds}{\displaystyle}



\newcommand{\ta}{\!\rightarrow\!}
\newcommand{\tr}{\!:\!}
\newcommand{\Tr}{:}

% канонические комбинаторы Y, S, K, I, etc
\newcommand{\canonComb}[1]{\boldsymbol{#1}}
% обычные комбинаторы fac, mult, etc
\newcommand{\comb}[1]{\mathtt{#1}} %% \mathsf

\newenvironment{MyList}[1][4pt]{
  \begin{enumerate}[0.]
  \setlength{\parskip}{0pt}
  \setlength{\itemsep}{#1}
}{       
  \end{enumerate}
}


\begin{document}

\thispagestyle{empty}

\begin{center}
\textbf{Курс: Функциональное программирование}

\textbf{Домашнее задание 2}

\textbf{Тигиной Марии 2 группа}
\end{center}

\bigskip

\bigskip




%-------------------------------------------------------------------------------------


\begin{MyList0}[8pt]

\item Приведите пример терма(замкнутого) который бы:
\begin{enumerate}[a)]
\item Находился в WHNF, но не в HNF.\\
$\l x \ar (\l y \ar yy) x$\\
(1 балл)\\

\item Находился бы в HNF, но не в NF.\\
$\l x \ar x ((\l y \ar y)x) $\\
(1 балл)\\
\end{enumerate}
\end{MyList0}

\begin{MyList}[8pt]

\item Напишите следующие функции над числами Чёрча:
\begin{enumerate}[a)]
\item Сравнения m <= n:\\
$le = \l m\; n\; \ar iszero\; (minus\; a\; b)$\\
(1 балл)

\item Сравнения m >= n:\\
$ge = \l m\; n\; \ar iszero\; (minus\; b\; a)$\\
(1 балл)

\item Сравнения m < n:\\
$lt = \l m\; n\; \ar le\; (succ\; a)\; b$\\
(1 балл)

\item Сравнения m > n:\\
$gt = \l m\; n\; \ar ge\; a\; (succ\; b)$\\
(1 балл)

\item Проверка на равенство m == n:\\
$equals = \l m\; n\; \ar and\; (le\; a\; b)\; (ge\; a\; b)$\\
(1 балл)

\item Функцию, суммирующую числа от 0 до n(комбинатором неподвижной точки пользоваться нельзя):\\
$sum = \l n\; \ar rec\; \l n\; (\l x\; y\; \ar plus\; x\; (succ\; y))\; \overline{1} $\\
(1 балл)

\end{enumerate}\\


\item Напишите предикат $isEven$, возвращающий $tru$, если его аргумент четное число и $fls$ – в противном случае. В этом задании нельзя использовать комбинатор неподвижной точки.\\
$isEven = \l m\; \ar rec\; m\; (\l x\; y\; \ar not\; x)\; tru$\\
(2 балла)\\

\item Напишите следующие функции над списками. Комбинатором неподвижной точки пользоваться нельзя
\begin{enumerate}[a)]

\item length – длина списка\\
$length = \l l \ar l\; (\l x\; \ar  succ)\; \overline{0} $\\
(1 балл)

\item sum – сумма элементов списка\\
$sum = \l l\; \ar (\l x\; \ar plus\; x)\; \overline{0}  $\\
(1 балл)

\item применяет функцию $f$ ко всем элементам списка $map succ [1, 2, 3] = [2, 3, 4]$\\
$map = \l f\; l \ar (\l x\; \ar cons\; (f\; x))\; nil$\\
(1 балл)

\item reverse – разворачивает список. $reverse [1,2,3] = [3,2,1]$\\
$push = \l x\; l \ar l\; cons\; (cons\; x\; nil)\;$\\
$reverse = \l l \ar l\; push\;  nil$\\
(2 балла)

\item tail – хвост списка. $tail [1, 2, 3] = [2, 3].$\\
Обозначим: $pair\;x\;y = (x,\; y)$\\
$tail = \l l \ar snd(l(\l x\; p \ar (cons\; x (fst\; p),\; fst(p))(nil,\; nil)))$\\
(2 балла)
\end{enumerate}\\

\item Используя комбинатор неподвижной точки найдите терм $F$ такой что:
\begin{enumerate}[a)]
\item Для любого $M$ было бы верно $F$ $M$ = $M$ $F$\\
$F\; M = (\l f\; x \ar x\; f) F\; M$\\
$F = (\l f\; x \ar x\; f)\; F $\\
$F = Y(\l f\; x \ar x\; f)$\\
(1 балла)

\item Для любых $M$ и $N$ было бы верно $F\; M\; N\; = N\; F\; (M\; N\; F)$\\
$F\; M\; N\; = (\l f\; x\; y \ar x\; f(x\; y\; f))F\; M\; N\;$\\
$F\; = (\l f\; x\; y \ar x\; f(x\; y\; f))\;F$\\
$F = Y(\l f\; x\; y \ar x\; f(x\; y\; f))$\\
(1 балла)

\end{enumerate}\\

\item Пусть $f$ и $g$ определены взаимно-рекурсивно:\\
$f=Ffg$\\
$g=Gfg$\\
Используя комбинатор неподвижной точки найдите нерекурсивные определения функций $f$ и $g$\\
$f=(\l x \ar F\;x\;g)f = Y(\l x \ar F\;x\;g)$\\
$g=(\l y \ar G\;f\;y)g = (\l y \ar G\;(Y(\l x \ar F\;x\;y))y)\;g =  Y(\l y \ar G\;(Y(\l x \ar F\;x\;y))y)$\\
(2 балла)

\item Докажите, что ваше определение $reverse$ является инволюцией для любого конечного списка. То есть $reverse\; (reverse\; xs)\; =\; xs$ для любого конечного списка $xs$.\\
Пусть $xs = cons\; a_1( cons\; a_2( cons\; a_3(...(cons\; a_n\; nil)..)$\\
Тогда после применения $xs\; push\; nill$ список станет иметь вид:\\
$reverse\; xs = push\; a_1( push\; a_2( push\; a_3(...(push\; a_n nil)..)$\\
Докажем, что такое выражение правда переворачивает нам список\\
База(для одного элемента):\\
$push\; a\; nil\\
= (\l x\; l \ar l\; cons\; (cons\; x\; nil))\; a\; nil \;\\
= nil\; cons\; (cons\; a\; nil)\\
= cons\; a\; nil$\\
Предположение:\\
Пусть $n - 1$ последних элементов мы удачно перевернули и получили список:\\
$xs' = push\; a_{2}( push\; a_{3}(...(push\; a_n\; nil)..) = cons\; a_{n}( cons\; a_{n-1}(...(cons\; a_2\; nil)..)$\\
Переход:\\
$xs = push\; a_1\; xs'\\
= (\l x\; l \ar l\; cons\; (cons\; x\; nil))\; a_1\; xs' \;\\
= xs'\; cons\; (cons\; a_1\; nil)\\
= cons\; a_{n}( cons\; a_{n-1}(...(cons\; a_1\; nil)..)$

(2 балла)

\item (Опциональное). Напишите терм, вычисляющий n-ное число Фибоначчи. В этом задании нельзя пользоваться комбинатором неподвижной точки.\\
$fib = \l n \ar snd (rec\; n\; (\l p x \ar (plus\; (fst\; p) (snd\; p), (fst\; p))) (\overline{1}, \overline{0}))$\\
(2 балла)

\end{MyList0}
\end{document}
